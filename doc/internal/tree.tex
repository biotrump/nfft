%              Tree -- a macro to make aligned (horizontal) trees in TeX
%
%    Input is of the form
%        \tree
%           item
%           \subtree
%              \leaf{item}
%                 .
%                 .
%                 .
%           \endsubtree
%           \subtree
%              .
%              .
%              .
%           \endsubtree
%        \endsubtree
%     \endtree
%
%    Nesting is to any level.  \leaf is defined as a subtree of one item:
% \def\leaf#1{\subtree#1\endsubtree}.
%
%    A structure:
%       \subtree
%          item_part1
%          item_part2
%               .
%               .
%               .
%
% will print item_part2 directly below item_part1 as a single item
% as if they were in a \box.
%
%    The macro is a 3-pass macro.  On the first pass it sets up a data
% structure from the \subtree ... \endsubtree definitions.  On the second pass
% it recursively calculates the width of each level of the tree.  On the third
% pass it sets up the boxes, glue and rules.
%
%    By David Eppstein, TUGboat, vol. 6 (1985), no. 1, pp. 31--35.
%    Transcribed by Margaret Kromer (peg), Feb., 1986.
%
%             Pass 1
% At the end of pass 1, the tree is coded as a nested collection of \hboxes
% and \vboxes.
\newbox\treebox\newcount\treeboxcnt
\def\tree{\message{Begin tree}\treeboxcnt=1\global\setbox\treebox=\boxtree}
\def\subtree{\ettext \advance\treeboxcnt by 1 \boxtree}
\def\leaf#1{\subtree#1\endsubtree}
\def\endsubtree{\ettext \egroup \advance\treeboxcnt-1{}%
             \ifnum\treeboxcnt=-1 \treeerrora\fi}
\def\endtree{\endsubtree \ifnum\treeboxcnt>0 \treeerrorb\fi%
             \settreesizes \typesettree \message{-- end tree}}
% Error messages for unbalanced tree
\def\treeerrora{\errhelp=\treeerrorahelp%
             \errmessage{Unbalanced tree -- too many endsubtrees}}
\newhelp\treeerrorahelp{There are more subtrees closed than opened}
\def\treeerrorb{\errhelp=\treeerrorbhelp%
             \errmessage{Unbalanced tree -- not enough endsubtrees}}
\newhelp\treeerrorbhelp{Not all the subtrees of the tree are closed.
If you continue, you'll get some mysterious secondary errors.}
%        Set up \vbox containing root of tree
\newif\iftreetext\treetextfalse         % Whether still aligning text
\def\boxtree{\hbox\bgroup               % Start outer box of tree or subtree
  \baselineskip 2.5ex                   % Narrow line spacing slightly
  \tabskip 0pt                          % No spurious glue in alignment
  \vbox\bgroup                          % Start inner text \vbox
  \treetexttrue                         % Remember for \ettext
  \let\par\crcr \obeylines              % New line breaks without explicit \cr
  \halign\bgroup##\hfil\cr}             % Start alignment with simple template
\def\ettext{\iftreetext                 % Are we still in inner text \vbox?
  \crcr\egroup \egroup \fi}             % Yes, end alignment and box
%             Pass 2
% Recursively calculate widths of tree with \setsizes; keep results in
% \treesizes; \treewidth contains total width calculated so far.  \treeworkbox
% is workspace containing subtree being sized.
\newbox\treeworkbox
\def\cons#1#2{\edef#2{\xmark #1#2}}     % Add something to start of list
\def\car#1{\expandafter\docar#1\docar}  % Take first element of list
\def\docar\xmark#1\xmark#2\docar{#1}    % ..by ignoring rest in expansion
\def\cdr#1{\expandafter\docdr#1\docdr#1}% Similarly, drop first element
\def\docdr\xmark#1\xmark#2\docdr#3{\def#3{\xmark #2}}
\def\xmark{\noexpand\xmark}             % List separator expands to self
\def\nil{\xmark}                        % Empty list is just separator
\def\settreesizes{\setbox\treeworkbox=\copy\treebox%
              \global\let\treesizes\nil \setsizes}
\newdimen\treewidth                     % Width of this part of the tree
\def\setsizes{\setbox\treeworkbox=\hbox\bgroup% Get a horiz list as a workspace
  \unhbox\treeworkbox\unskip            % Take tree, unpack it into horiz list
  \inittreewidth                        % Get old width at this level
  \sizesubtrees                         % Recurse through all subtrees
  \sizelevel                            % Now set width from remaining \vbox
  \egroup}                              % All done, finish our \hbox
\def\inittreewidth{\ifx\treesizes\nil   % If this is the first at this level
    \treewidth=0pt                      % ..then we have no previous max width
 \else \treewidth=\car\treesizes        % Otherwise take old max level width
   \global\cdr\treesizes                % ..and advance level width storage
   \fi}                                 % ..in preparation for next level.
\def\sizesubtrees{\loop                 % For each box in horiz list (subtree)
  \setbox\treeworkbox=\lastbox \unskip  % ..pull it off list and flush glue
  \ifhbox\treeworkbox \setsizes         % If hbox, it's a subtree - recurse
  \repeat}                              % ..and loop; end loop on tree text
\def\sizelevel{%
  \ifdim\treewidth<\wd\treeworkbox      % If greater than previous maximum
  \treewidth=\wd\treeworkbox \fi        % Then set max to new high
 \global\cons{\the\treewidth}\treesizes}% In either case, put back on list
%             Pass 3
% Recursively typeset tree with \maketree by adding an \hbox containing
% a subtree (in \treebox) to the horizontal list.
\newdimen\treeheight                    % Height of this part of the tree
\newif\ifleaf                           % Tree has no subtrees (is a leaf)
\newif\ifbotsub                         % Bottom subtree of parent
\newif\iftopsub                         % Top subtree of parent
\def\typesettree{\medskip\maketree\medskip}  % Make whole tree
\def\maketree{\hbox{\treewidth=\car\treesizes  % Get width at this level
  \cdr\treesizes                        % Set up width list for recursion
  \makesubtreebox\unskip                % Set \treebox to text, make subtrees
  \ifleaf \makeleaf                     % No subtrees, add glue
  \else \makeparent \fi}}               % Have subtrees, stick them at right
{\catcode`@=11                          % Be able to use \voidb@x
\gdef\makesubtreebox{\unhbox\treebox    % Open up tree or subtree
  \unskip\global\setbox\treebox\lastbox % Pick up very last box
  \ifvbox\treebox                       % If we're already at the \vbox
    \global\leaftrue \let\next\relax    % ..then this is a leaf
  \else \botsubtrue                     % Otherwise, we have subtrees
    \setbox\treeworkbox\box\voidb@x     % Init stack of processed subs
    \botsubtrue \let\next\makesubtree   % ..and call \maketree on them
  \fi \next}}                           % Finish up for whichever it was
\def\makesubtree{\setbox1\maketree      % Call \maketree on this subtree
  \unskip\global\setbox\treebox\lastbox % Pick up box before it
  \treeheight=\ht1                      % Get height of subtree we made
  \advance\treeheight 2ex               % Add some room around the edges
  \ifhbox\treebox \topsubfalse          % If picked up box is a \vbox,
    \else \topsubtrue \fi               % ..this is the top, otherwise not
  \addsubtreebox                        % Stack subtree with the rest
  \iftopsub \global\leaffalse           % If top, remember not a leaf
    \let\next\relax \else               % ..(after recursion), set return
    \botsubfalse \let\next\makesubtree  % Otherwise, we have more subtrees
  \fi \next}                            % Do tail recursion or return
\def\addsubtreebox{\setbox\treeworkbox=\vbox{\subtreebox\unvbox\treeworkbox}}
\def\subtreebox{\hbox\bgroup            % Start \hbox of tree and lines
  \vbox to \treeheight\bgroup           % Start \vbox for vertical rules
    \ifbotsub \iftopsub \vfil           % If both bottom and top subtree
        \hrule width 0.4pt              % ..vertical rule is just a dot
     \else \treehalfrule \fi \vfil      % Bottom gets half-height rule
    \else \iftopsub \vfil \treehalfrule % Top gets half-height the other way
     \else \hrule width 0.4pt height \treeheight \fi\fi % Middle, full height
    \egroup                             % Finish vertical rule \vbox
  \treectrbox{\hrule width 1em}\hskip 0.2em\treectrbox{\box1}\egroup}
\def\treectrbox#1{\vbox to \treeheight{\vfil #1\vfil}}
\def\treehalfrule{\dimen\treeworkbox=\treeheight   % Get total height
  \divide\dimen\treeworkbox 2%
  \advance\dimen\treeworkbox 0.2pt      % Divide by two, add half horiz height
  \hrule width 0.4pt height \dimen\treeworkbox}% Make a vertical rule that high
\def\makeleaf{\box\treebox}             % Add leaf box to horiz list
\def\makeparent{\ifdim\ht\treebox>%
    \ht\treeworkbox                     % If text is higher than subtrees
    \treeheight=\ht\treebox             % ..use that height
  \else \treeheight=\ht\treeworkbox \fi % Otherwise use height of subtrees
  \advance\treewidth-\wd\treebox        % Take remainder of level width
  \advance\treewidth 1em                % ..after accounting for text and glue
  \treectrbox{\box\treebox}\hskip 0.2em % Add text, space before connection
\treectrbox{\hrule width \treewidth}%
  \treectrbox{\box\treeworkbox}}        % Add \hrule, subs
% No idea what \spouse is supposed to do... wasn't included
\def\spouse{\bf}
