%\begin{savequote}[8cm]
%  ``Confession: I played golf. Confession: I hired a cleaning lady. Confession: I am owner of the Turbo nose-hair trimmer with optional ear-hair accessory.''
%  \qauthor{Dan Zevin}
%\end{savequote}
%\makeatletter

\chapter{Introduction}
\label{Introduction}

The \emph{fast Fourier transform (FFT)} has become one of the most important 
and widely used algorithms today. Cooley and Tukey (\cite{cotu}) published
and publicized the first description of a fast algorithm for computing certain
trigonometric sums, generally referred to as the \emph{discrete} or 
\ emph{finite Fourier transform}, in 1965. In fact, developments since then now
let us speak of a whole class of algorithms. The original Culey-Tukey FFT
used a recursive scheme to split one large transform successively into
smaller transforms achieving an asymptotical complexity of 
$\bigo{N \log N}$ \emph{floating point operations (flops)} instead of
$\bigo{N^2}$ for a direct computation, where $N$ is the transform length. 
Efficient and highly optimized implementations also for the
multidimensional case are available (\cite{fftw}).
Surprisingly, the idea of Cooley and Tukey was already described by Gauss around
1805 (\cite{gauss},\cite{hejobu}) interpolating the trajectories of the asteroids Pallas 
and Juno. But his work was not widely recognized and only published posthumously.
The wide areas of applications nowadays using FFT-techniques, including 
time-frequency analysis, signal-processing and the numerical solution of 
partial differential equations, underline the great importance and impact of fast
DFT algorithms.
Recently, algorithms have been developed for a more general type of discrete 
transform, namely the \emph{nonuniform discrete Fourier transform (NDFT)}. 
While the DFT is a bijective and easily invertible mapping of $N$ 
ingoing coefficients to $N$ outgoing coefficients corresponding to
uniformly distributed 'nodes' on an interval, the NDFT generalizes
discrete Fourier sums for arbitrary node distributions, hence the name
'nonuniform'. Fast algorithms have been described in several papers 
(\cite{bey95},\cite{duro93},\cite{fesu02},\cite{four},\cite{Ja},\cite{Pe},
\cite{scsc},\cite{ware98}) and a C subroutine library is available 
(\cite{kupo02C}).
A different line of generalization leaves the 'classical' setting of 
Fourier series on a multidimensional torus and describes analogously
Fourier series on different geometries like the surface of a
multidimensional sphere. Particularly, the unit sphere $\twosphere$ 
embedded into $\R^3$ has practical relevance in quite a wide range of 
applications, in many cases due to its correspondence to the surface 
of the earth. Fields of interest are numerical analysis in
geo-sciences, for example in the computation of climate models for
weather forecasts, or regarding the gravitational potential of the 
earth. Moreover, data distributed on the surface of a sphere arise
in many other applications in a natural way. Examples are
molecular dynamics for protein docking problems and 
gamma-cameras for applications in computed tomography.
Unfortunately, the spherical setting differs from the 'classical' 
one in that the numerical treatment of many computational problems
is quite more challenging. Also there exists a rich theory for 
analysis on the sphere, fast algorithms allowing for efficient and
reliable discrete transforms in terms of the so-called 
\emph{spherical harmonics}, the spherical counterpart of the 
Fourier basis $\set{e^{\im k x}}_{k \in \Z}$ on the interval,
haven't been developed until a first paper by Driscoll and Healy. 
During the past years, major progress has been made and several 
different techniques have been developed. The common idea is to
transform the spherical discrete Fourier transformation into a
'classical' two-dimensional discrete Fourier transform. The
different proposed algorithms mainly differ in the ansatz chosen
to compute the transformation efficiently. A careful analysis
of numerical instabilities involved has proven to be of major 
importance.
The aim of this text is to give a short introduction to Fourier 
series on the sphere $\twosphere$ and to describe a particular
algorithms with more detail. 